\documentclass[5pt]{resume}
\usepackage[margin=0.35in]{geometry} % for changing margins
\usepackage{color}
\usepackage[dvipsnames]{xcolor}
\usepackage{hyperref}
\setlength{\voffset}{-.75in}

\definecolor{titleColor}{HTML}{7e97ac}
\definecolor{default}{HTML}{585858}

\topmargin=0.0in 
% \oddsidemargin=0.0in
% \evensidemargin=0.0in 
% \textwidth=7.25in 
% \marginparwidth=0.5in
\headheight=0pt 
\headsep=0pt
\textheight=11.0in

\pagenumbering{gobble} % remove page numbers
\setcounter{tocdepth}{0}  
\usepackage{titlesec} % this is for changing title formatting
\titleformat*{\subsubsection}{\large\bfseries} 

\begin{document}
\name{ Thomas DuPlessis }
\shortcontact{thomas.p.duplessis@gmail.com}{(516)-457-1174}{\url{dupes.dev}}{\url{github.com/ThomasDuplessis}}


{\color{titleColor}\section{Skills}}
\noindent\textbf{Best Languages:} {\color{default} \verb!C++!, Java, Go }\\
\textbf{Other Languages:} {\color{default} Haskell, C, Python, shell, javascript, typescript} \\
\textbf{Technologies:} {\color{default} MapReduce, Flume, SQL, Git, Linux, \LaTeX } 

{\color{titleColor} \section{Work Experience}}

\subsection{\textbf{Software Engineer - L4, Google \hfill June 2017-present}}
{\color{default} Google Maps data infrastructure, Geo-Feeds: developing 
  distributed pipelines to process third party (and internal) data into google 
  maps. Allows clients to access google maps infrastucture to match their data 
  to internal data, obtain statistics on the quality of their data, and splice
  their data before submitting to the map.
  Worked on all parts of system:
  \begin{itemize}
	\itemsep0em 
  	\item optimizing core parts of the infrastructure.
  	\item creating a highly configurable alerting system visibile from our UI that allowed clients to make their own filters.
	\item implementing API's that internal teams can use to access our processed data.
\end{itemize}
}

\subsection{\textbf{Software Engineering intern, Google \hfill Summer 2016}}
{\color{default} Google Maps data infrastructure, Geo-Feeds. Created a new stats collection process that informed clients which of their data matched to what data in the production database.}

\subsection{\textbf{Summer Technology Analyst, Citi \hfill Summer 2015}} 
{\color{default} Worked on team developing a front office
  platform using Java and Spring. I added JSON support to the backend trade
  processing system so that JSON encoded trade information can be sent through
  an API call. I also made a stress test program for our backend using Gatling
  and Scala.}

\subsection{\textbf{Software Engineering Intern, Kongsberg ITS \hfill Summer 2014}}
{\color{default} Worked on a digital radio system: doing socket programming in
  \verb!C++! as well as GUI development in \verb!C#! for a military vehicle control system. I
  wrote testing software for the control system as well in \verb!C++! and \verb!C#! that
  communicated over a CAN bus.  }


{\color{titleColor}\section{Education}}
\subsection{\textbf{Stony Brook University }}
{\color{default}MS Computer Science} \hfill \textbf{2017}\\
{\color{default}BS Computer Science, BS Applied Mathematics and Statistics} \hfill \textbf{2016} \\ 
\textbf{GPA}: {\color{default} 3.59} 
\textbf{Awards}: {\color{default}Dean’s List, University Scholars program, Presidential Scholarship}


{\color{titleColor}\section{Projects/Open Source Contributions}}
\subsection{\textbf{Visual SLAM Robotics Implementation (course project)}}
{\color{default} Final project for graduate Computer Vision course on a team of two. We
implemented a version of the SLAM algorithm for a real robot based off the
paper: \href{
  ''http://www.robots.ox.ac.uk/~cmei/articles/AConstantTimeEfficientStereoSLAMSystem_rss_09.pdf''}{A
  Constant Time Stereo Slam} which details an efficient method to localize and
map a robot's environment through live a video feed using \verb!C++! and OpenCV (Code
located on github).}
\subsection{\textbf{Emacs-Eclim (open source project)}}
{\color{default} Contributed to the open source project “Emacs-Eclim”
  which provides an interface for emacs to use Eclim, a backend code completion
  and project manager backend for text editors, using Eclipse. I helped add
  Scala completion and error checking to the Emacs Eclim project. }
\subsection{\textbf{Friend Finder App (Android app)}} 
{\color{default} Mhacks 2013 project where we built an android app that
  would point in the direction of another connected android.}
\\
\end{document}
